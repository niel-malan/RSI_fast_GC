% Title:    A LaTeX Template For Responses To a Referees' Reports
% Author:   Petr Zemek <s3rvac@gmail.com>
% Homepage: https://blog.petrzemek.net/2016/07/17/latex-template-for-responses-to-referees-reports/
% License:  CC BY 4.0 (https://creativecommons.org/licenses/by/4.0/)
\documentclass[10pt]{article}

% Allow Unicode input (alternatively, you can use XeLaTeX or LuaLaTeX)
\usepackage[utf8]{inputenc}
\usepackage{amsmath}
\usepackage{eurosym}
\usepackage{textcomp}

\usepackage{microtype,xparse,tcolorbox}
\newenvironment{reviewer-comment }{}{}
\tcbuselibrary{skins}
\tcolorboxenvironment{reviewer-comment }{empty,
  left = 1em, top = 1ex, bottom = 1ex,
  borderline west = {2pt} {0pt} {black!20},
}
\ExplSyntaxOn
\NewDocumentEnvironment {response} { +m O{black!20} } {
  \IfValueT {#1} {
    \begin{reviewer-comment~}
      \setlength\parindent{2em}
      \noindent
      \ttfamily #1
    \end{reviewer-comment~}
  }
  \par\noindent\ignorespaces
} { \bigskip\par }

\NewDocumentCommand \Reviewer { m } {
  \section*{Comments~by~Reviewer~#1}
}
\ExplSyntaxOff
\AtBeginDocument{\maketitle\thispagestyle{empty}\noindent}

% You can get probably get rid of these definitions:
\newcommand\meta[1]{$\langle\hbox{#1}\rangle$}
\newcommand\PaperTitle[1]{``\textit{#1}''}

\title{Response letter of \meta{RSI19-AR-01677} \\
  Based on the Referees' Report}
\author{D. Malan \and S.J. van der Walt \and E.R. Rohwer}
\date{\today}

\begin{document}

\noindent
\textbf{Chin-Tu Chen}\\
\textbf{Associate Editor}\\
\textbf{Review of Scientific Instruments}

\medskip

\begin{quote}
Revision of the paper \meta{RSI19-AR-01677} submitted to the Review of
Scientific Instruments, entitled \PaperTitle{A high-repetition-rate, fast
temperature programmed gas chromatograph and its on-line coupling to a
supercritical fluid chromatograph (SFC×GC)}, based on the referees' report.
\end{quote}

\medskip

\noindent
Dear Chin-Tu Chen: 

Thank you for your email dated 7 November 2019 enclosing the reviewers’ comments. We
have carefully reviewed the comments and have revised the manuscript accordingly. Our
responses are given in a point-by-point manner below. 

We hope the revised version is now suitable for publication and look forward to hearing from you
in due course.

\vspace{\baselineskip}

\noindent
Yours sincerely,\\
D. Malan\\
Department of Chemistry\\
University of Pretoria


\Reviewer{\#1}
\begin{response}{Abstract, last line: the sentence fragment 'separating fatty information" needs revision.
	}
 	The sentence was revised and clarified.
\end{response}

\begin{response}{p3, orthogonal and general elution problem. In trying to wrap
my head around this argument, I can't help but think that despite their comment
about GC×GC wrap-around, it does seem that this would be even more of a problem
in their current setup or those using LC-GC. The authors miss the opportunity to
justify the need for fast temperature programmed GC in their second dimension.
Perhaps laying their arguments out in more detail here would be useful.
} 
The problem with the vagueness of the argument was identified. The paragraphs
were changed to clarify how the general elution problem applies to
multidimensional chromatography, and to emphasize that SFC×GC demands
temperature programming in the GC dimension.
\end{response}

\begin{response}{p6 last line, sentence structure needs revision.
	}  The sentence was revised and clarified.
\end{response}

\begin{response}{p7, description of the connection blocks. Refer the reader to see Figure 1 here. 
	}  A reference to Figure 1 was added.
\end{response}

\begin{response}{Also, singular-plural confusion in :"These blocks fulfilled three roles: It (1) sealed ... "
	}  
	The confusion was cleared up. 
\end{response}

\begin{response}{Figure 1. It is not clear from the drawing that the cryogenic
coolant actually enters the heater tube. While the dimensions of the drawing are
perhaps selected to be according to scale, the reader would benefit from a
little artistic liberty here. (g) "electrical connection" is a little vague.
	} The figure was changed to add a schematic diagram that illustrate the coolant
	flow, and the caption was clarified.
\end{response}

\begin{response}{p9. Reference resistor and ballast resistor. What are the
dimensions of the wires used to construct these resistors? What is the
resistance of the ballast? 
	} 
	The requested details were added, as well as a figure that shows the construction of the relevant resistors.
\end{response}

\begin{response}{p13, A. Heating uniformity. Text says video is shown in
Figure3. Figure 3 is a screen capture of the video shown in Supplemental
Material. Correct accordingly.
	}
	We recognize the inaccuracy, and the text was corrected. 
\end{response}

\begin{response}{ p13. Gradients by cooling. Not sure how this statement
accurately reflects the repeated heating cycles shown in Figure 4. Further
information and a more detailed description of Figure 4 is required. Does the
difference between the two calibration points along the length of the column (33
and 66 cm) increase with temperature? 
	}  
	The discussion on the gradients measured by thermocouple was revised and expanded for greater clarity. 
\end{response}

\begin{response}{Figure 5. Explain in the heading the blue and red lines.
	}  
	The caption was revised with a better description that includes an explanation of the colours.
\end{response}

\begin{response}{Figure 2. Can be improved by showing more details of the
circuit, rather than just three resistors in series.
	}  The figure was replaced by a more detailed one, and the description in the text was adapted to match the figure.   
\end{response}

\begin{response}{A few references appear incomplete. e.g. 17 and 29.}  
	The bibliography was reviewed and all the incomplete references were updated.
\end{response}

\Reviewer{\#2}

\begin{response}{p1 Abstract, sentence change: We demonstrate the fast
chromatograph by separating fatty acid methyl esters by carbon number, which
would be useful in the food and biodiesel industries.
	}  
	The sentence was clarified. 
\end{response}

\begin{response}{p2: This concept is called comprehensively coupled
chromatography. For example, in comprehensively coupled gas chromatography
(GC×GC). Confirm correct nomenclature in as published by Marriot in LCGC 2012.
	} We thank the reviewers for drawing attention to using correct nomenclature. We
	revised the manuscript to ensure the recommended nomenclature was used throughout.
\end{response}

\begin{response}{p2 Sentence change: The use of narrow-bore columns and thin
films reduce run times by improving mass transfer between the mobile and
stationary phases, resulting in analytes spending more time in the mobile phase
allowing shorter retention times.

	} We agree that the sentence needed clarification, but disagree with the
	explanation. Improved mass transfer does not allow analytes to spend more time
	in the mobile phase, but it does reduce peak broadening.  The sentence was
	revised thusly: \textit{Typically, short columns and high carrier flows rates
	are used, which reduce retention times by reducing void times. Resolution can be
	maintained at even higher gas flows by using narrow-bore columns and thin films:
	these improve the mass transfer rate between the mobile and stationary phases,
	which reduces peak broadening by bringing the analyte exchange between the
	phases closer to equilibrium at higher-than-optimum carrier gas velocities.}

\end{response}

\begin{response}{p3 sentence change: it becomes unlikely that one set of
acceptable operating conditions (temperature, flow and stationary phase) will
give acceptable (fast enough and with adequate resolution) separation in both
dimensions.
	}  This sentence was revised as part of general improvement in the argument.
	(See Reviewer 1, comment ``p.3'')
\end{response}

\begin{response}{p3 sentence change: so late that the next modulation period has
already started before its elution and the peak only appears on the next 2D
chromatogram 
	}  The sentence was revised as suggested.
\end{response}

\begin{response}{p3 sentence change: If wraparound is a problem in GC×GC, it
becomes inevitable in SFC×GC since the first dimension (SFC) does not normally
separate compounds in a way that correlates with vapour pressure.
	}  This sentence was improved by revising the paragraph. 
\end{response}

\begin{response}{p5 sentence change: The Zip ScienticTM solution operates by forced convection.}  
	The sentence was changed to the suggested one. 
\end{response}

\begin{response}{p5: sentence change: In our case we applied it as a fast second
dimension separation, comprehensively coupled to a supercritical fluid
chromatograph.
	}  The sentence was changed to the suggested one. 
\end{response}

\begin{response}{p6 clarification: clarify what type of static restriction was
used.
	}  The description of the static restrictor was improved. 
\end{response}

\begin{response}{p6: sentence change: the stop valve opens and the SFC eluate
exits the restrictor into the hot splitless inlet of the GC, where it evaporates
and is transported onto the GC column.
	}  This sentence was improved by rewriting the paragraph. 
\end{response}

\begin{response}{p6: sentence change: which of course means the start of the
temperature program, which desorbs the analyteS trapped on the head of the cold
column.
	}  This sentence was improved by rewriting the paragraph.
\end{response}

\begin{response}{p7: sentence change: The split vent valve was disconnected from
the gas chromatograph CONTROL and CONNECTED TO the control computer described
below 
	}  The sentence was changed to the suggested one. 
\end{response}

\begin{response}{p10: Figure 2: Figure do not contain that much detail and may
be omitted.
	}  The figure was replace by a more detailed one, and the text adapted to match
	the figure. (See Reviewer 1, comment ``Figure 2'')
\end{response}

\begin{response}{p12: abbreviation: LabVIEW 7.1TM (National Instruments). This
abbreviation was used earlier p10 and should be clarified there.
	}  We recognize the oversight and corrected it.
\end{response}

\begin{response}{p15: sentence change: In the instrument presented here, THE
power was limited by the voltage output range of the DAC of the data acquisition
board, and not by the power of the heater.
	}  The sentence was replaced by the suggested one.
\end{response}


\end{document}

