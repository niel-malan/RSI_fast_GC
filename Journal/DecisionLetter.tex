
Email
Manuscript #	RSI19-AR-01677
Title	A high-repetition-rate, fast temperature programmed gas chromatograph and its on-line coupling to a supercritical fluid chromatograph (SFC×GC)
Corresponding Author	Egmont R Rohwer (University of Pretoria Department of Chemistry)
Date:	07-Nov-2019 19:42:33
Last Sent:	07-Nov-2019 19:42:33
Created By:	Redacted
From:	rsi-edoffice@aip.org
To:	egmont.rohwer@up.ac.za
Subject:	RSI: MS #RSI19-AR-01677 Decision Letter
Email	Dear Prof. Rohwer

Thank you for submitting your manuscript, referenced below, to Review of Scientific Instruments for consideration.

"A high-repetition-rate, fast temperature programmed gas chromatograph and its on-line coupling to a supercritical fluid chromatograph (SFC×GC)"

RSI19-AR-01677

Your manuscript has been reviewed, and I am pleased to inform you that it has received a favorable recommendation from the reviewers. Please see the report below, which lists the comments of the reviewers and myself. It contains suggestions for minor revisions that I believe will improve your paper.

Please make the appropriate changes and upload your revised manuscript files using the link below. When uploading your revised manuscript, you should:

- Indicate in your response letter how the manuscript has been revised to
address the points raised by the reviewers. For comments or suggestions with
which you do not agree, please provide a rebuttal.

- Include your manuscript number in your response letter.

- Upload your revised manuscript file in .doc, .docx, or .tex format with no
highlighted or marked text. (You will have the option to upload a separate,
marked-up version of your manuscript.)

- Upload separate figure files for each cited figure in EPS (preferred), PS,
TIFF (.tif), PDF, or JPEG (.jpg) format. For multi-panel figures, please upload
a single file containing all figure panels.

Click on the following link to submit your revised manuscript:

Link Not Available

Your revised manuscript is due 06-Jan-2020. Please submit your revised manuscript promptly. A manuscript returned after 06-Jan-2020 will be regarded as a new submission and will be assigned a new submission date. If that is required, please refer to your previous submission (include the previous manuscript number) in the cover letter.

Sincerely,

Chin-Tu Chen
Associate Editor
Review of Scientific Instruments



------------------------------------------------------------------------------
Manuscript #RSI19-AR-01677:

Editor's Comments:
Please review and consider carefully the critiques provided and issues raised by the two reviewers, and modify and improve your manuscript accordingly.
When you resubmit your revised paper, please include your point-to-point response to these critiques and issues.
-- CT Chen


Reviewer Comments:
Reviewer #1 Evaluations:
Recommendation: Publish after minor revision
Technically sound: Y
New ideas: Y
Just a variation of known device or technique: N
Appropriate journal: Y
Proper context with related work: Y
Clear explanation: Y
Adequate references: Y
Suitable title: Y
Adequate abstract: Y
Significant numerical quantities: Y
Clear figures with captions: Y
Excessive text or figures: Y
English satisfactory: Y
Regular Article vs Note: Regular Article

Reviewer #1 (Remarks):

The manuscript describes the design and construction of a novel high-repetition rate, resistively-heated, and actively-cooled programmed gas chromatograph, and its implementation as a second dimension in comprehensive multidimensional chromatography. The chromatographic arrangement is different from typical comprehensive separations in that the polar separation precedes the volatility separation. The authors do an excellent job of introducing the topic to a non-specialist audience, almost to a fault, and the manuscript could perhaps be a little more concise throughout.
While the GC is described as part of an SFCxGC system, its design would lend itself to other applications such as portable and fast GC for example in in-situ analyses.
While the design is described in sufficient detail, in some instances additional information could be provided and a few other corrections are suggested below.

Abstract, last line: the sentence fragment 'separating fatty information" needs revision.

p3, orthogonal and general elution problem. In trying to wrap my head around this argument, I can't help but think that despite their comment about GCxGC wrap-around, it does seem that this would be even more of a problem in their current setup or those using LC-GC. The authors miss the opportunity to justify the need for fast temperature programmed GC in their second dimension. Perhaps laying their arguments out in more detail here would be useful.

p6 last line, sentence structure needs revision.

p7, description of the connection blocks. Refer the reader to see Figure 1 here. Also, singular-plural confusion in :"These blocks fulfilled three roles: It (1) sealed ... "
Figure 1. It is not clear from the drawing that the cryogenic coolant actually enters the heater tube. While the dimensions of the drawing are perhaps selected to be according to scale, the reader would benefit from a little artistic liberty here.
(g) "electrical connection" is a little vague.

p9. Reference resistor and ballast resistor. What are the dimensions of the wires used to construct these resistors? What is the resistance of the ballast?

p13, A. Heating uniformity. Text says video is shown in Figure3. Figure 3 is a screen capture of the video shown in Supplemental Material. Correct accordingly.
p13. Gradients by cooling. Not sure how this statement accurately reflects the repeated heating cycles shown in Figure 4. Further information and a more detailed description of Figure 4 is required. Does the difference between the two calibration points along the length of the column (33 and 66 cm) increase with temperature?

Figure 5. Explain in the heading the blue and red lines.

Figure 2. Can be improved by showing more details of the circuit, rather than just three resistors in series.

A few references appear incomplete. e.g. 17 and 29.


Reviewer #2 Evaluations:
Recommendation: Publish after minor revision
Technically sound: Y
New ideas: Y
Just a variation of known device or technique: N
Appropriate journal: Y
Proper context with related work: Y
Clear explanation: Y
Adequate references: Y
Suitable title: Y
Adequate abstract: Y
Significant numerical quantities: Y
Clear figures with captions: Y
Excessive text or figures: Y
English satisfactory: Y
Regular Article vs Note: Regular Article

Reviewer #2 (Remarks):

The highly repetitive fast GC for the use in 2D separations is very valuable. The use of this in SFCxGC allows for a the application and testing of a wide variety of column and selectivity combinations. The ease of column change in the 2nd dimension becomes important during method development, however, changing columns for varying selectivity in the 1st dimension (SFC) can be done easily. The authors have gone to great lengths to ensure that the heating, cooling and other aspects are repeatable and fast. The combination with SFC, as in this paper, will allow for a wide variety of application areas.

Proposed changes:
p1 Abstract, sentence change: We demonstrate the fast chromatograph by separating fatty acid methyl esters by carbon number, which would be useful in the food and biodiesel industries.

p2: This concept is called comprehensively coupled chromatography. For example, in comprehensively coupled gas chromatography (GCxGC). Confirm correct nomenclature in as published by Marriot in LCGC 2012.

p2 Sentence change: The use of narrow-bore columns and thin films reduce run times by improving mass transfer between the mobile and stationary phases, resulting in analytes spending more time in the mobile phase allowing shorter retention times.

p3 sentence change: it becomes unlikely that one set of acceptable operating conditions (temperature, flow and stationary phase) will give acceptable (fast enough and with adequate resolution) separation in both dimensions.

p3 sentence change: so late that the next modulation period has already started before its elution and the peak only appears on the next 2D chromatogram

p3 sentence change: If wraparound is a problem in GCxGC, it becomes inevitable in SFCxGC since the first dimension (SFC) does not normally separate compounds in a way that correlates with vapour pressure.

p5 sentence change: The Zip ScienticTM solution operates by forced convection.

p5: sentence change: In our case we applied it as a fast second dimension separation, comprehensively coupled to a supercritical fluid chromatograph.

p6 clarification: clarify what type of static restriction was used.

p6: sentence change: the stop valve opens and the SFC eluate exits the restrictor into the hot splitless inlet of the GC, where it evaporates and is transported onto the GC column.

p6: sentence change: which of course means the start of the temperature program, which desorbs the analyteS trapped on the head of the cold column.

p7: sentence change: The split vent valve was disconnected from the gas chromatograph CONTROL and CONNECTED TO the control computer described below

p10: Figure 2: Figure do not contain that much detail and may be omitted.

p12: abbreviation: LabVIEW 7.1TM (National Instruments). This abbreviation was used earlier p10 and should be clarified there.

p15: sentence change: In the instrument presented here, THE power was limited by the voltage output range of the DAC of the data acquisition board, and not by the power of the heater.



------------------------------------------------------------------------------
