% Title:    A LaTeX Template For Responses To a Referees' Reports
% Author:   Petr Zemek <s3rvac@gmail.com>
% Homepage: https://blog.petrzemek.net/2016/07/17/latex-template-for-responses-to-referees-reports/
% License:  CC BY 4.0 (https://creativecommons.org/licenses/by/4.0/)
\documentclass[10pt]{article}

% Allow Unicode input (alternatively, you can use XeLaTeX or LuaLaTeX)
\usepackage[utf8]{inputenc}
\usepackage{amsmath}
\usepackage{eurosym}
\usepackage{textcomp}

\usepackage{microtype,xparse,tcolorbox}
\newenvironment{reviewer-comment }{}{}
\tcbuselibrary{skins}
\tcolorboxenvironment{reviewer-comment }{empty,
  left = 1em, top = 1ex, bottom = 1ex,
  borderline west = {2pt} {0pt} {black!20},
}
\ExplSyntaxOn
\NewDocumentEnvironment {response} { +m O{black!20} } {
  \IfValueT {#1} {
    \begin{reviewer-comment~}
      \setlength\parindent{2em}
      \noindent
      \ttfamily #1
    \end{reviewer-comment~}
  }
  \par\noindent\ignorespaces
} { \bigskip\par }

\NewDocumentCommand \Reviewer { m } {
  \section*{Comments~by~Reviewer~#1}
}
\ExplSyntaxOff
\AtBeginDocument{\maketitle\thispagestyle{empty}\noindent}

% You can get probably get rid of these definitions:
\newcommand\meta[1]{$\langle\hbox{#1}\rangle$}
\newcommand\PaperTitle[1]{``\textit{#1}''}

\title{Response letter of \meta{RSI19-AR-01677} \\
  Based on the Referees' Report}
\author{D. Malan \and S.J. van der Walt \and E.R. Rohwer}
\date{\today}

\begin{document}

\noindent
\textbf{Chin-Tu Chen}\\
\textbf{Associate Editor}\\
\textbf{Review of Scientific Instruments}

\medskip

\begin{quote}
Revision of the paper \meta{RSI19-AR-01677} submitted to the Review of
Scientific Instruments, entitled \PaperTitle{A high-repetition-rate, fast
temperature programmed gas chromatograph and its on-line coupling to a
supercritical fluid chromatograph (SFC×GC)}, based on the referees' report.
\end{quote}

\medskip

\noindent
Dear Chin-Tu Chen: 

Thank you for your email dated 30 January 2020 enclosing the reviewers’ comments. We
have carefully reviewed the comments and have revised the manuscript accordingly. Our
responses are given in a point-by-point manner below. 

We hope the revised version is now suitable for publication and look forward to hearing from you
in due course.

\vspace{\baselineskip}

\noindent
Yours sincerely,\\
D. Malan\\
Department of Chemistry\\
University of Pretoria


\Reviewer{\#1}
\begin{response}{will [give an] acceptable}
 	The sentence was corrected.
\end{response}

\begin{response}{middle paragraph. The sentence construction is complicated with the nested parentheses.
} 
The parentheses were removed and a sentence providing more detail was added.
\end{response}

\begin{response}{Also the text describes a micro union brazed onto the block,
but the figure shows a drilled and tapped hole.
}  Although the relevant figure was intended to be conceptual in nature, we
agree that a difference between the text and the figure can be confusing. The
figure was changed to match the text.
\end{response}

\begin{response}{p11 digital to analog converter converter . converter is repeated. 
	}  The repeated word was removed. 
\end{response}

\begin{response}{p15. Chromatographically, would it be better if the warmer side of the column was towards the detector or towards the inlet?
	}  
	A sentence and a citation was added to show a possible benefit if the warmer end of the column was towards the inlet.
\end{response}


\end{document}

